%\documentclass[a4paper,10pt]{book}

\documentclass[ebook,oneside]{memoir}
% https://tex.stackexchange.com/questions/1551/use-latex-to-produce-epub
% tex4ebook -f mobi filename.tex

\usepackage[utf8x]{inputenc}

%\usepackage{polyglossia}
%\setdefaultlanguage{spanish}
\usepackage[spanish]{babel}

\usepackage{amsmath, amssymb, amsthm, epsf, amsfonts, thmtools, mathtools}
\usepackage{cool}
\usepackage{physics} % Paquete para derivadas
\usepackage[colorlinks]{hyperref}
\usepackage{xpatch}
 
\usepackage{graphicx}
\graphicspath{{figures/}} % Carpeta de figura, para escribir texto usar \caption{text}

\usepackage{tikz-cd}
\usetikzlibrary{babel}
% \usepackage{pgfplots}
\usetikzlibrary{decorations.markings,arrows, shapes.geometric}
% Permite añadir símbolos de relación centrados en lugar de flechas
\tikzset{
	symbol/.style={
		draw=none,
		every to/.append style={
			edge node={node [sloped, allow upside down, auto=false]{$#1$}}}
	}
}

\usepackage{fancyhdr}
\pagestyle{fancy}
\lhead[\thepage]{\rightmark}
\rhead[\leftmark]{\thepage}
\cfoot[]{}

\DeclareMathOperator{\sub}{\subset}
\DeclareMathOperator{\subeq}{\subseteq}

\renewcommand{\phi}{\varphi}
\renewcommand{\chaptername}{Tema}

% Definimos proposiciones, lemas y demas
\newtheorem{thm}{Teorema}[chapter]
\newtheorem{cor}[thm]{Corolario}
\newtheorem{lem}[thm]{Lema}
\newtheorem{prop}[thm]{Proposición}
\newtheorem{defn}[thm]{Definición}
\newtheorem{rem}[thm]{Observaciones}
\newtheorem{eje}[thm]{Ejemplos}
\newtheorem{ejercicio}{Ejercicio}[chapter]

% Podemos dar una lista de las definiciones principales y demás
\renewcommand{\listtheoremname}{Lista de Teoremas}
\xpatchcmd{\listoftheorems}{\@fileswfalse}{}{\typeout{yes}}{\typeout{no}}

\newcommand{\Restrict}[2]{{% we make the whole thing an ordinary symbol
		\left.\kern-\nulldelimiterspace{}% automatically resize the bar with \right
		#1 % the function
		\vphantom{\big|} % pretend it's a little taller at normal size
		\right|_{#2} % this is the delimiter
}} % Source: https://tex.stackexchange.com/questions/22252


\newenvironment{subproof}[1][\proofname]{%
	\renewcommand{\qedsymbol}{$\blacksquare$}%
	\begin{proof}[#1]%
	}{%
	\end{proof}%
}

\title{Archivo genérico}
\author{Jesús Camacho Moro}
\date{\today}

\begin{document}
	
\maketitle

\tableofcontents

\listoftheorems[ignore=thm]

\chapter{Nuevos comandos}
Ahora el comando $\backslash$\text{varphi} escribe la función $\phi$. 

\section{Diagramas}
Para insertar un diagrama de composición de funciones:

\begin{tikzcd}
	A \arrow[rd] \arrow[r, "\phi"] & B \\
	& C
\end{tikzcd}


\section{Fórmulas}

\begin{thm}\label{vínculo}
Podemos escribir parciales con el comando $\dv{f(x)}{x}$ o bien $\pdv{f}{y}{x}$
\end{thm}

\begin{proof}
Una prueba se puede dividir en varios:
\begin{subproof}[$\implies$]
Una dirección.
\end{subproof}
\begin{subproof}[$\impliedby$]
La otra dirección usando el propio teorema \ref{vínculo}.
\end{subproof}
\end{proof}

\begin{figure}[h]
	\centering
	\begin{tikzcd}
		U \arrow[r, ""] \arrow[d, ""']
		& W \arrow[d, ""] \\
		\varphi(U) \arrow[r, ""]
		& \psi(W)
	\end{tikzcd}
	\caption{Texto}
	\label{fig:texto}
\end{figure}




\begin{thebibliography}{}
	
\bibitem[T]{t} N. Apellido. \textit{Titulo.} Edición , Editorial.
	
\end{thebibliography}

\end{document}